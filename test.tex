\documentclass[12pt]{article}

\usepackage{xltxtra, setspace}

%% fonts

% xeCJK options from &lt;http://mesokosmos.blogger.de/stories/1818274/&gt;:
\usepackage[%
  boldfont,
  CJKnumber,
  CJKchecksingle
]{xeCJK}

% main font for CJK input from &lt;http://www.fontpark.net/en/font/kozuka-gothic-pro-m/&gt;
\setCJKmainfont[Script=CJK]{Yu Mincho}
% \setCJKmainfont[Script=CJK]{Kozuka Mincho Pro}

% version of the above font with rotated characters for vertical text,
%  font needs to come with ``vrt2'' feature
\setCJKfamilyfont{cjk-vert}[Script=CJK,RawFeature=vertical]{Yu Mincho}
% \setCJKfamilyfont{cjk-vert}[Script=CJK,RawFeature=vertical]{Kozuka Mincho Pro}

% main font for non-CJK input from &lt;http://www.exljbris.com/delicious.html&gt;

%% Furigana

% use \ruby{kanji}{kana} to set Furigana
\usepackage[CJK,overlap]{ruby}
% position of Furigana: below/right
\renewcommand{\rubysep}{-4.3ex}
% increase line space for Furigana
\onehalfspacing

%% alignment

% also from &lt;http://mesokosmos.blogger.de/stories/1818274/&gt;:
\XeTeXlinebreaklocale &quot;ja&quot;
\XeTeXlinebreakskip=0em plus 0.1em minus 0.01em
% we also drop paragraph indentation in order to get proper alignment
\setlength{\parindent}{0pt}

\begin{document}

\section*{Regular English Text}

This is \textbf{English} text and should be \textit{set} in Delicious
(or \textsc{Delicious SmallCaps}).

\section*{Regular Japanese Text}

きのう、ラースは \ruby{山}{やま}\ruby{本}{もと}を あいました。 
「こんにちは、\ruby{山}{やま}\ruby{本}{もと}さん。おげんき ですか。」

Note the nice alignment of the letters, in spite of the 「」 brackets.
(Some combinations of punctuation might break the alignment, though.)

\section*{Vertical Japanese Text}

\begin{figure}[h!]
\begin{center}
\rotatebox{-90}{
\begin{minipage}{0.35\textwidth}
\CJKfamily{cjk-vert}
\onehalfspacing % needs to be repeated here (?!)
きのう、\textbf{ラース}は \ruby{山}{やま}\ruby{本}{もと}を あいました。 
「こんにちは、\ruby{山}{やま}\ruby{本}{もと}さん。おげんき ですか。」
\end{minipage}
}
\end{center}
\end{figure}

That's quite a nice alignment of the letters, in spite of the 「」 brackets
and the (artificially created) bold letters. (Some combinations of punctuation
might break the alignment, though.)

\end{document}
